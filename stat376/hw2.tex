
\documentclass[11pt]{article}
\usepackage[paper=letterpaper, margin=.5in]{geometry}
\pdfpagewidth 8.5in
\pdfpageheight 11in
\setlength\parindent{0in} %%% Packages
% First four - AMS (american mathematical society). General math goodness. I use the align* enviorment in particular
% multirow, multicol allow for certain kinds of tables
% enumerate lets you determine the style of the counter for the enumerate enviorment
% graphicx lets you include pictures
% listings lets you stick in blocks of code
% placeins defines "\FloatBarrier", which stops tables from moving around
\usepackage{amsmath, amscd, amssymb, amsthm, multirow, multicol, enumerate, graphicx, listings, placeins} 
\newcommand{\Z}{\mathbb{Z}}
\newcommand{\R}{\mathbb{R}}
\newcommand{\Q}{\mathbb{Q}}
\newcommand{\C}{\mathbb{C}}
\newcommand{\N}{\mathbb{N}}
\newcommand{\V}{\mathbb{V}}
\newcommand{\U}{\mathcal{U}}
\newcommand{\del}{\partial}
\newcommand{\real}{\textrm{Re }}
\newcommand{\imag}{\textrm{Im }}
\newcommand{\pd}[2]{\frac{\partial #1}{\partial #2}}
\newcommand{\deriv}[2]{\frac{d #1}{d #2}}
\newcommand{\sumk}{\sum_{k=1}^\infty}
\newcommand{\sumj}{\sum_{j=1}^\infty}
\newcommand{\sumn}{\sum_{n=0}^\infty}
\newcommand{\summ}[2]{\sum_{k=#1}^{#2}}
\newcommand{\sig}[1]{\sum_{#1 =1}^\infty}
\newcommand{\un}[1]{\bigcup_{#1 =1}^\infty}
\newcommand{\inter}[1]{\bigcap_{#1 =1}^\infty}
\newcommand{\ip}[2]{\langle #1, #2 \rangle}
\newcommand{\ipxu}{\langle x,u_j \rangle}
\newcommand{\uj}{\{u_j\}_{j=1}^\infty}
\newcommand{\B}{\mathcal{B}}

\newcommand{\E}{\mathrm{E}}
\newcommand{\var}{\mathrm{Var}}
\newcommand{\cov}{\mathrm{Cov}}
\newcommand{\ST}{mbox{ s.t. }}

\newcommand{\Example}{\noindent {\bf Example. \quad} }
\newcommand{\Proof}{\noindent {\bf Proof: \quad} }
\newcommand{\Remark}{\noindent {\bf Remark. \quad} }
\newcommand{\Remarks}{\noindent {\bf Remarks. \quad} }
\newcommand{\Case}{\noindent {\underline{Case} \quad} }

\newcommand{\st}{ \; \big | \:}

\newcommand{\deuc}{d_{\mathrm euc}}
\newcommand{\dtaxi}{d_{\mathrm taxi}}
\newcommand{\ddisc}{d_{\mathrm disc}}
\newtheorem{theorem}{Theorem}[section]
\newtheorem{lemma}[theorem]{Lemma}
\newtheorem{proposition}[theorem]{Proposition}
\newtheorem{corollary}[theorem]{Corollary}
\theoremstyle{definition}
\newtheorem{definition}[theorem]{Definition}
\newtheorem{example}[theorem]{Example}

\begin{document}
%%%%%%%%%%%%%%%%%%%%%%%%%%%%%%%%%%%%%%%%%%%%%%%%%%%%%%%%%%%%%%%%%%%%%%%%%%%%%%%%%%%%%%%%%%%%%%%%%%%%%%%%%%%%%%%%%%%%%%%%%%%%%%%%%%%%%
STAT 376 Homework 2 \hfill Aaron Maurer
\vspace{2mm}
\hrule
\vspace{2mm}

\begin{itemize}
    \item[1.]
        \begin{itemize}
            \item[(a)]
                Its easy to see that an optimal clustering is not necessary unique. For instance, if our points are in $\R^2$, drawn uniformly from $\{(0,0),(0,1),(1,0),(1,1)\}$m, then the optimal risk for two clusters is clearly $R(C^*)=.25$. However, this risk can be achieved by two different $\hat C$, \(\{(.5,0),(.5,1)\}\) or \(\{(0,.5),(1,.5)\}\), making $C^*$ not unique.
            \item[(b)]
                Let $C^*_{k}$ achieve the minimal risk $R^{(k)}$. Now let $C_{k+1}=C^*_{k}\cup\{c_{k+1}\}$ for some \(c_{k+1}\) in the space of $X$. For all $X$, 
                \begin{align*}
                    \Rightarrow& \forall c_i \in C_k^*\; :\; \min_{c_j\in C_{k+1}}\|X-c_j\| \leq \|X-c_i\|  \\
                    \Rightarrow& \min_{c_j\in C_{k+1}}\|X-c_j\| \leq \min_{c_i\in C_{k}} \|X-c_i\|  \\
                    \Rightarrow& R(C^*_{k}) \geq R(C_{k+1})  \\
                \end{align*}
                Since, by definition, \(R(C^*_{k}) =R^{(n)}\) and \(R(C_{k+1}) \geq R^{(n+1)}\), we have
                \[R^{(n)} = R(C^*_{k}) \geq R(C_{k+1}) \geq R^{(n+1)}\]
                So $R^{(n)}$ is non increasing in $n$.

        \end{itemize}
\end{itemize}

\end{document}
