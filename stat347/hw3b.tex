
\documentclass[11pt]{article}
\usepackage[paper=letterpaper, margin=.5in]{geometry}
\pdfpagewidth 8.5in
\pdfpageheight 11in
\setlength\parindent{0in}

%%% Packages
% First four - AMS (american mathematical society). General math goodness. I use the align* enviorment in particular
% multirow, multicol allow for certain kinds of tables
% enumerate lets you determine the style of the counter for the enumerate enviorment
% graphicx lets you include pictures
% listings lets you stick in blocks of code
% placeins defines "\FloatBarrier", which stops tables from moving around
\usepackage{amsmath, amscd, amssymb, amsthm, multirow, multicol, enumerate, graphicx, listings, placeins} 
\newcommand{\Z}{\mathbb{Z}}
\newcommand{\R}{\mathbb{R}}
\newcommand{\Q}{\mathbb{Q}}
\newcommand{\C}{\mathbb{C}}
\newcommand{\N}{\mathbb{N}}
\newcommand{\V}{\mathbb{V}}
\newcommand{\U}{\mathcal{U}}
\newcommand{\del}{\partial}
\newcommand{\real}{\textrm{Re }}
\newcommand{\imag}{\textrm{Im }}
\newcommand{\pd}[2]{\frac{\partial #1}{\partial #2}}
\newcommand{\deriv}[2]{\frac{d #1}{d #2}}
\newcommand{\sumk}{\sum_{k=1}^\infty}
\newcommand{\sumj}{\sum_{j=1}^\infty}
\newcommand{\sumn}{\sum_{n=0}^\infty}
\newcommand{\summ}[2]{\sum_{k=#1}^{#2}}
\newcommand{\sig}[1]{\sum_{#1 =1}^\infty}
\newcommand{\un}[1]{\bigcup_{#1 =1}^\infty}
\newcommand{\inter}[1]{\bigcap_{#1 =1}^\infty}
\newcommand{\ip}[2]{\langle #1, #2 \rangle}
\newcommand{\ipxu}{\langle x,u_j \rangle}
\newcommand{\uj}{\{u_j\}_{j=1}^\infty}
\newcommand{\B}{\mathcal{B}}

\newcommand{\E}{\mathrm{E}}
\newcommand{\var}{\mathrm{Var}}
\newcommand{\cov}{\mathrm{Cov}}
\newcommand{\ST}{mbox{ s.t. }}

\newcommand{\Example}{\noindent {\bf Example. \quad} }
\newcommand{\Proof}{\noindent {\bf Proof: \quad} }
\newcommand{\Remark}{\noindent {\bf Remark. \quad} }
\newcommand{\Remarks}{\noindent {\bf Remarks. \quad} }
\newcommand{\Case}{\noindent {\underline{Case} \quad} }

\newcommand{\st}{ \; \big | \:}

\newcommand{\deuc}{d_{\mathrm euc}}
\newcommand{\dtaxi}{d_{\mathrm taxi}}
\newcommand{\ddisc}{d_{\mathrm disc}}
\newtheorem{theorem}{Theorem}[section]
\newtheorem{lemma}[theorem]{Lemma}
\newtheorem{proposition}[theorem]{Proposition}
\newtheorem{corollary}[theorem]{Corollary}
\theoremstyle{definition}
\newtheorem{definition}[theorem]{Definition}
\newtheorem{example}[theorem]{Example}

\newcommand{\hwhead}[1]{#1 \hfill Aaron Maurer \vspace{2mm} \hrule \vspace{2mm}}

\begin{document}
%%%%%%%%%%%%%%%%%%%%%%%%%%%%%%%%%%%%%%%%%%%%%%%%%%%%%%%%%%%%%%%%%%%%%%%%%%%%%%%%%%%%%%%%%%%%%%%%%%%%%%%%%%%%%%%%%%%%%%%%%%%%%%%%%%%%%
\hwhead{STAT 347 Homework 3b}
\begin{itemize}
    \item[1.]
        \begin{itemize}
            \item[(a)] \& (b) \\
                I've plotted the given rates and the fits below. The fits are from a model with the formulae \(chem +dose\), where the log of dose was not taken.
                \begin{center}
                    \includegraphics[width=12cm]{hw3/A1_ab} 
                \end{center}
            \item[(c)]
                We want to compare to logistic models, one with the formulae \(chem + ldose\), where the linear predictor for each chemical is identical except for different intercepts, and \(chem*ldose\), where the linear predictor for each chemical is linear in log dose with different intercept and different slope. Since the first model is fit on a subset of the second, we can use a likelihood ratio test to evaluate whether the larger model is significantly better. In this case, with a p-value of .18, we can not conclude so, and default to the smaller model being superior.
            \item[(d)]
                These two models define the same subspace, but in slightly different ways, with the second omitting a constant. The coefficients on $ldose$ will be identical, while the coefficients on $chem$ in the second model will be those in the first plus the intercept from before. If $\Sigma_1$ and $\Sigma_2$ are the covariance matrices, then
                \[\left[\begin{array}{cccc} 1 & 0 & 0 & 0 \\
                                            1 & 1 & 0 & 0 \\
                                            1 & 0 & 1 & 0 \\
                                            0 & 0 & 0 & 1 \\
                           \end{array}\right] \Sigma_1
                   \left[\begin{array}{cccc} 1 & 0 & 0 & 0 \\
                                             1 & 1 & 0 & 0 \\
                                             1 & 0 & 1 & 0 \\
                                             0 & 0 & 0 & 1 \\
                           \end{array}\right] = \Sigma_2 \]
                The matrix is, of course, the linear transformation to turn the first space into the second.       
            \item[(e)]
                I for the potency of the combo relative to DDT, I got an estimate of $2.45$ with a $90\%$ CI of \((2.12,2.89)\). Relative to BHC, I got an estimate of $3.08$ with a CI of \((2.64,3.67)\).
            \item[(f)]
                Fitting the models with alternate links, we only seem to get a mildly better fit with the log-log link, raising the log-likelihood from -43.0 to -42.0. All the others have lower (worse) log-likelihoods. When we estimate the potency of the combo relative to the individual chemicals, for DDT I got an estimate of 2.37 with a CI of \((2.07,2.74)\) and for BHC I got an estimate of 3.05 with a CI of \((2.63,3.63)\).
            \item[(g)]
                We will have a 99\% kill rate for the combo with dose $x$ if 
                \begin{align*}
                    .99 &= \frac{1}{1-e^{-\alpha_{combo} - \beta x}} \\
                    \log\left(\frac{.99}{.01}\right) &= \alpha_{combo} + \beta x \\
                    4.59 &= \alpha_{combo} + \beta x \\
                    x &= \frac{4.59-\alpha_{combo}}{\beta} \\
                \end{align*}
                We can estimate a CI on this by fitting a model with an offset of $-4.59$, and then using Fieller's method on the coefficients for the combo and the log dose. Doing so gives us an estimate of $10.59$ with a CI of \((7.17,17.35)\).
            \item[(h)] From our analysis, a few things are clear. For one, the combination dose is more effective than either of the individual insecticides, having much higher potency than each of them. Additionally, from the high significance on dosage, we can see more insecticide kills more bugs, at least up to a point, but the effect of increasing the dosage doesn't change much or at all between insecticides. Also, DDT seems to be somewhat stronger than BHC when each are used on their own.
        \end{itemize}
    \item[2.]
        I fit a log-linear Poisson model predicting the total number of children ($mean \times sample size$) based on the three given variables with an offset of $\log(sample size)$. Log-linear is appropriate since we're fitting counts, and the weighting should account for the varying sample size. A Poisson model seems appropriate here since the variation will increase with the expected number of children (mostly due to more mothers in the sample). A quasi-Poission model might also be appropriate, but when one fits such a model, the dispersion parameter isn't that much smaller at $.8$. Since the data is, if anything, under dispersed, worst case is the model has conservative confidence intervals. \vspace{2mm} \\
        The variables included were Education, a factor indicating no education, lower elementary, upper elementary, or secondary and up, years since marriage in five year buckets as a factor, and whether the mother lived in a rural area or not. I used the model formula $Years + Education + Rural$. Both dropping education, the least significant of the factors, and interacting years and rural, the most significant variables, resulted in inferior models by the likelihood ratio test. Fertility seemed to be heavily effected by years since first marriage and rural setting. The highest education levels was also highly significant, but the differences between the other three levels was marginal or non-existent. \vspace{2mm}\\
        The model gives a prediction that an urban woman with upper elementary education will have 3.60 children on average after 10 years of marriage, with a $95\%$ CI of \((3.34,3.89)\) \vspace{2mm} \\
        To get the lifetime average prediction for a rural women with secondary education, I predicted for the oldest group, for whom 25+ years had passed since there first marriage. This ignores that some women wouldn't live this long, driving up the average, and that some women in this group might still have children, driving down the average. None the less, with no other obvious way to answer this question, the prediction is 6.02 children with a $90\%$ CI of \((5.42,6.69)\).

\end{itemize}

\end{document}
