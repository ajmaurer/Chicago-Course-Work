
\documentclass[11pt]{article}
\usepackage[paper=letterpaper, margin=.5in]{geometry}
\pdfpagewidth 8.5in
\pdfpageheight 11in
\setlength\parindent{0in}

%%% Packages
% First four - AMS (american mathematical society). General math goodness. I use the align* enviorment in particular
% multirow, multicol allow for certain kinds of tables
% enumerate lets you determine the style of the counter for the enumerate enviorment
% graphicx lets you include pictures
% listings lets you stick in blocks of code
% placeins defines "\FloatBarrier", which stops tables from moving around
\usepackage{amsmath, amscd, amssymb, amsthm, multirow, multicol, enumerate, graphicx, listings, placeins} 
\newcommand{\Z}{\mathbb{Z}}
\newcommand{\R}{\mathbb{R}}
\newcommand{\Q}{\mathbb{Q}}
\newcommand{\C}{\mathbb{C}}
\newcommand{\N}{\mathbb{N}}
\newcommand{\V}{\mathbb{V}}
\newcommand{\U}{\mathcal{U}}
\newcommand{\del}{\partial}
\newcommand{\real}{\textrm{Re }}
\newcommand{\imag}{\textrm{Im }}
\newcommand{\pd}[2]{\frac{\partial #1}{\partial #2}}
\newcommand{\deriv}[2]{\frac{d #1}{d #2}}
\newcommand{\sumk}{\sum_{k=1}^\infty}
\newcommand{\sumj}{\sum_{j=1}^\infty}
\newcommand{\sumn}{\sum_{n=0}^\infty}
\newcommand{\summ}[2]{\sum_{k=#1}^{#2}}
\newcommand{\sig}[1]{\sum_{#1 =1}^\infty}
\newcommand{\un}[1]{\bigcup_{#1 =1}^\infty}
\newcommand{\inter}[1]{\bigcap_{#1 =1}^\infty}
\newcommand{\ip}[2]{\langle #1, #2 \rangle}
\newcommand{\ipxu}{\langle x,u_j \rangle}
\newcommand{\uj}{\{u_j\}_{j=1}^\infty}
\newcommand{\B}{\mathcal{B}}

\newcommand{\p}{\mathrm{P}}
\newcommand{\E}{\mathrm{E}}
\newcommand{\var}{\mathrm{Var}}
\newcommand{\cov}{\mathrm{Cov}}
\newcommand{\ST}{mbox{ s.t. }}

\newcommand{\Example}{\noindent {\bf Example. \quad} }
\newcommand{\Proof}{\noindent {\bf Proof: \quad} }
\newcommand{\Remark}{\noindent {\bf Remark. \quad} }
\newcommand{\Remarks}{\noindent {\bf Remarks. \quad} }
\newcommand{\Case}{\noindent {\underline{Case} \quad} }

\newcommand{\st}{ \; \big | \:}

\newcommand{\deuc}{d_{\mathrm euc}}
\newcommand{\dtaxi}{d_{\mathrm taxi}}
\newcommand{\ddisc}{d_{\mathrm disc}}
\newtheorem{theorem}{Theorem}[section]
\newtheorem{lemma}[theorem]{Lemma}
\newtheorem{proposition}[theorem]{Proposition}
\newtheorem{corollary}[theorem]{Corollary}
\theoremstyle{definition}
\newtheorem{definition}[theorem]{Definition}
\newtheorem{example}[theorem]{Example}

\newcommand{\hwhead}[1]{#1 \hfill Aaron Maurer \vspace{2mm} \hrule \vspace{2mm}}

\begin{document}
%%%%%%%%%%%%%%%%%%%%%%%%%%%%%%%%%%%%%%%%%%%%%%%%%%%%%%%%%%%%%%%%%%%%%%%%%%%%%%%%%%%%%%%%%%%%%%%%%%%%%%%%%%%%%%%%%%%%%%%%%%%%%%%%%%%%%
\hwhead{STAT 347 Homework 3, Part 1}
\begin{itemize}
    \item[0.]
        \begin{itemize}
            \item[(i)]
                We can calculate the joint distribution easily from the description:
                \[\p(Y,N) = \begin{cases} 
                              \;            \frac{1}{3} &\mbox{if } Y=0,N=0 \\
                              \;   \frac{1}{3}(1-\pi_1) &\mbox{if } Y=0,N=1 \\
                              \;       \frac{1}{3}\pi_1 &\mbox{if } Y=1,N=1 \\
                              \;   \frac{1}{3}(1-\pi_2) &\mbox{if } Y=0,N=2 \\
                              \;       \frac{1}{3}\pi_2 &\mbox{if } Y=2,N=2 \\
                              \;            0           &\mbox{otherwise } \\
                            \end{cases} \]
            \item[(ii)]
                $Y/N \st N=1$ and $Y/N \st N=2$ are both Bernoulli random variables with probability with probabilities $\pi_1$, $\pi_2$, so 
                \[\E(Y/N \st N=1) = \pi_1,\; \var(Y/N \st N=1) = \pi_1(1-\pi_1),\;\E(Y/N \st N=2) = \pi_2,\; \var(Y/N \st N=1) = \pi_2(1-\pi_2) \]
                If $\pi_1=\pi_2$, $Y/N \st N=1 \sim Y/N \st N=2$, so $Y/N$ and $N$ are uncorrelated.
            \item[(iii)]
                Since the wells are independent, $Y\sim B(26,\pi_1)+2\times B(15,\pi_2)$, so
                \[ \E[Y] = \E[B(26,\pi_1)] + 2\E[B(15,\pi_2)] = 26\pi_1 + 30\pi_2\]
                and
                \[ \var(Y) = \var[B(26,\pi_1)] + 4 \var[B(15,\pi_2)] = 26\pi_1(1-\pi_1) + 60\pi_2(1-\pi_2)\]
            \item[(iv)]
                Fitting this distribution $F$ to the data, with $m=26\times1 + 15\times 2 = 56$, we would estimate 
                \[ \mu = m\pi = \E[Y] = 26\pi_1 + 30\pi_2 \Longrightarrow \pi = \frac{26\pi_1 + 30\pi_2}{56}\] 
                so we would calculate $\sigma^2$ as
                \begin{align*}
                    \sigma^2m\pi(1-\pi) &= \var[y] \\
                    \sigma^256\frac{26\pi_1 + 30\pi_2}{56}\left(1-\frac{26\pi_1 + 30\pi_2}{56}\right) &= 26\pi_1(1-\pi_1) + 60\pi_2(1-\pi_2)\\
                    \sigma^2&= \frac{56(26\pi_1(1-\pi_1) + 60\pi_2(1-\pi_2))}{(26\pi_1 + 30\pi_2)(56-26\pi_1 - 30\pi_2)}\\
                \end{align*}
                Which, in the case where $\pi_1=\pi_2=\pi$, reduces to 
                \[ \sigma^2 = \frac{76}{56} = \frac{19}{14}\]
            \item[(v)]
                Let $Y_1$ be the number of homogamic matings in single-mating wells out of $M_1$ total,  and $Y_2$ and $M_2$ the same for two mating wells. If we assume that $\pi_1=\pi_2=\pi$, then 
                \[ Y_2/2 \sim B(M_2/2,\pi) \quad \&\quad Y_1 \sim B(M_1,\pi)\]
                so 
                \[\var(Y_2) = \var(2B(M_2/2,\pi)) = 2M_2\pi(1-\pi)\]
                and 
                \[\var(Y_1) = \var(B(M_1,\pi)) = M_1\pi(1-\pi)\]
                so, if $2M_1=2M_2=M$,
                \begin{align*}
                    \var(Y) &= 2M_2\pi(1-\pi)+M_1\pi(1-\pi) \\
                    \sigma^2M\pi(1-\pi) &= \frac{3}{2}\pi(1-\pi) \\
                    \sigma^2 &= \frac{3}{2} \\
                \end{align*}
            \item[(vi)]
                This mechanism can't explain the chi-square test, which indicated under dispersion. If it was the case that one couldn't see one homogamic and one heterogamic mating in one well as we simulated, then the result is higher variance than the binomial model anticipates. This is the opposite of what is observed in the actual data set, suggesting this mechanism is not the explanation.  
            \item[(vii)]
                They would have concluded that the data does not follow a multinomial distribution by looking at the variance of frequencies among replicated experiments, for instance through the same chi square test as above. Since these frequencies had much lower variance than one would anticipate with a binomial model, they concluded the frequencies weren't binomial. This would also stand to reason, since with only two flies of each gender in a well, one pair mating would necessarily effect the probability of the other pair mating, invalidating independence among the mating events. The data not being binomial would invalidate the standard errors and p-values, which were made under an assumed binomial mode.
            \item[(viii)] By only analyzing the first mating in each well, they were assuring there was no dependence among mating events they counted. Since there is independence between wells, this first mating must necessarily be independent of the other counted matings. With this fix, the binomial model should hold again, allowing for simpler inference. 
        \end{itemize}
\end{itemize}

\end{document}
